% vim: set spell spelllang=en_us:
\documentclass[10pt, final]{article}
% \documentclass[11pt]{amsart}

\input{~/common}
% \input{~/globals/macros}

\hyphenation{auto-morphism auto-morphisms homo-geneous}

%%%%%%%%%%%%%%%%%%%%%%%%%%%%%%%%%%%%%%%%%%%%%%%%%%%%%%%%%%%%%%%%%%%%%%
%%%%%%%%%%%%%%%%%%%%%%%%%%%%%%%%%%%%%%%%%%%%%%%%%%%%%%%%%%%%%%%%%%%%%%
%%%%%%%%%%%%%%%%%%%%%%%%%%%%%%%%%%%%%%%%%%%%%%%%%%%%%%%%%%%%%%%%%%%%%%

% \title{%
%   On the derivations and automorphisms\\
%   of the algebra $\kk\langle x,y\rangle/(yx-xy-x^N)$
% }
% \author{Amittai Siavava\thanks{%
%     Undergraduate student at Dartmouth College.
%   }
% }
% \ifoptionfinal
%   {\date{\today}}
%   {\date{Started on March 2021; compiled \today}}

  
\begin{document}

% \setlength{\headheight}{13.0pt}
% \setlength{\footskip}{13.0pt}


% TITLE
\newdate{due-date}{28}{02}{2024}
\PSET{8 --- \displaydate{due-date}}
  {Winter 2024}
  {Erchenko}
  {Amittai Siavava}
  {Math 63: Real Analysis}

% CREDIT STATEMENT
\CreditStatement{
  I worked on these problems alone,
  with reference to class notes and the following books:
  \begin{enumarabic}
    \item \textbf{\textit{Introduction to Analysis}} by Maxwell Rosenlicht
  \end{enumarabic}
}

% \bigskip

\begin{problem}
  Prove that $\displaystyle \int\limits_0^1 f(x) \d x = 0$ if
  $f(\frac{1}{n})= 1$ for all $n \in \N$ and $f(x) = 0$ for all other $x$.
\end{problem}

% \begin{answer}
\begin{claim}
  $f$ is Rieman integrable on $[0, 1]$.
\end{claim}

\begin{proof}
  Since $f = 0$ at all points $x \in [0, 1] \setminus \{1/n : n \in \N\}$,
  $f$ is continuous at all such $x$. Therefore, we can consider
  the points of the form $1/n, n \in \N$ as discontinuities of $f$.
  However, since $\N$ has measure zero (since it is countable),
  $\set{1/n : n \in \N}$ also has measure zero.
  By the Lebesgue criterion for Riemann integrability, $f$ is Riemann integrable,
  so $\displaystyle \int\limits_0^1 f(x) \d x$ exists.
\end{proof}

\begin{claim}
  $\displaystyle \int\limits_0^1 f(x) \d x = 0$.
\end{claim}
\begin{proof}
  Given a partition $P_n$ of $[0, 1]$ into $n$ subintervals,
  \blue{\[
    L(f, P) = \sum_{i=1}^n m_i \Delta x_i
            \leq \int \limits_0^1 f(x) \d x
            \leq \sum_{i=1}^n M_i \Delta x_i
            = U(f, P),
  \]}
  where $m_i = \inf\set{f(x) : x \in [x_{i-1}, x_i]}$ and
  $M_i = \sup\set{f(x) : x \in [x_{i-1}, x_i]}$.
  Furthermore;
  \begin{equation}
    L(f, P) = \sum_{i=1}^n m_i \Delta x_i
            = \sum_{i=1}^n 0 \cdot \Delta x_i
            = 0
  \end{equation}
  and
  \begin{equation}
    U(f, P) = \sum_{i=1}^n M_i \Delta x_i.
  \end{equation}
  Since the set of points where $f$ is nonzero has measure zero,
  as we make the partitions finer and finer, $M_i \Delta x_i$ will
  either be zero or approach $0$ for all $i$, so $U(f, P)$ will also approach $0$.
  On the other hand, $L(f, P)$ will always be $0$.
  Thus, we can make $U(f, P) - L(f, P)$ arbitrarily small,
  so $\displaystyle \int\limits_0^1 f(x) \d x = 0$.
\end{proof}
  % negative space
  % \vspace{-1.5em}
% \end{answer}


\newpage
\begin{problem}
  Prove that if $f$ is a continuous real-valued function on the interval
  $[a, b]$ such that $f(x) \geq 0$ for all $x \in [a, b]$ and
  $f(x) > 0$ for some $x \in [a, b]$, then $\displaystyle \int\limits_a^b f(x) \d x > 0$.
\end{problem}

% \begin{answer}
  \begin{claim}
    $\displaystyle \int\limits_a^b f(x) \d x > 0$.
  \end{claim}  
  \begin{proof}
    First, note that $f$ is Riemann integrable on $[a, b]$ since it is continuous
    on $[a, b]$ and $[a, b]$ is a closed interval.
    We are given that $f(x) > 0$ for some $x \in [a, b]$.
    Since $f$ is continuous over $[a, b]$,
    for every $\epsilon > 0$ there exists a $\delta > 0$
    such that for all $x_1, x_2 \in [a, b]$, if $|x_1 - x_2| < \delta$
    then $\abs{f(x_1) - f(x_2)} < \epsilon$.
    Let $y = \inf\set{f(x) \given x \in [a, b]}$, with $\xi$ as the corresponding
    value for $x$. Since $f > 0$ at some point over $[a, b]$, $y > 0$.
    Pick $\epsilon = y/2$, and pick $\delta$ as above.
    Then, for all $x \in [\xi - \delta, \xi + \delta] \cap [a, b]$,
    $f(x) > y - \epsilon = y/2 > 0$.
    Therefore,
    \[ \text{Therefore, }\int\limits_{\max (a, \xi - \epsilon)}^{\min(b, \xi + \epsilon)} f(x) \d x > 0. \]
    Since $f(x) \geq 0$ for all $x \in [a, b]$, and $[a, b]$ is a superset
    of $[\max (a, \xi - \epsilon), \min(b, \xi + \epsilon)]$, it follows that
    \blue{
      \[
        \int\limits_a^b f(x) \d x \geq
        \int\limits_{\max (a, \xi - \epsilon)}^{\min(b, \xi + \epsilon)} f(x) \d x
        > 0.
      \]
    }
    Therefore, $\displaystyle \int\limits_a^b f(x) \d x > 0$.
  \end{proof}
% \end{answer}


\newpage
\begin{problem}
  Let $f : \R \to \R$ with
    \[
      f(x) = \begin{cases}
        0 & \text{ if } x \in \R \setminus \Q \text{ (i.e. not rational).}\\
        \frac{1}{q} & \text{ if } x \colonequals \frac{p}{q} \in \Q
        \text{ with $p, q$ coprime and $q > 0$}.
      \end{cases}
    \]
  Show that $\displaystyle \int\limits_0^1 f(x) \d x$ exists and is equal to $0$.\\
  \underline{Hint:} Use the Lebesgue criterion for integrability.
    In particular, you need to determine at what points $f$ is continuous.
\end{problem}

% \begin{answer}
  \begin{claim}
    $\displaystyle \int\limits_0^1 f(x) \d x$ exists.
  \end{claim}
  \begin{proof}
    We are given that $f$ is only nonzero for rational numbers of the form $p/q$,
    where $p, q \in \N$ and $p, q$ are coprime.
    Since the rational numbers are countable, $\Q$ has Lebesgue measure zero,
    meaning that the set of points where $f$ is nonzero,
    which is a subset of $\Q$, also has measure zero.
    By the Lebesgue criterion, $f$ is Riemann integrable on $[0, 1]$
    since the set of points where $f$ is discontinuous has measure zero,
    so $\displaystyle \int\limits_0^1 f(x) \d x$ exists.
  \end{proof}

  \begin{claim}
    $\displaystyle \int\limits_0^1 f(x) \d x = 0$.
  \end{claim}
  \begin{proof}
    Given a partition $P_n$ of $[0, 1]$ into $n$ subintervals,
    \blue{\[
      L(f, P) = \sum_{i=1}^n m_i \Delta x_i
              \leq \int \limits_0^1 f(x) \d x
              \leq \sum_{i=1}^n M_i \Delta x_i
              = U(f, P),
    \]}
    where $m_i = \inf\set{f(x) : x \in [x_{i-1}, x_i]}$ and
    $M_i = \sup\set{f(x) : x \in [x_{i-1}, x_i]}$.
    Furthermore;
    \begin{equation}
      L(f, P) = \sum_{i=1}^n m_i \Delta x_i
              = \sum_{i=1}^n 0 \cdot \Delta x_i
              = 0
    \end{equation}
    and
    \begin{equation}
      U(f, P) = \sum_{i=1}^n M_i \Delta x_i.
    \end{equation}
    Since the set of points where $f$ is nonzero has measure zero,
    as we make the partitions finer and finer, $M_i \Delta x_i$ will
    approach $0$ for all $i$, so $U(f, P)$ will also approach $0$
    (since only a smaller subset will have nonzero $M_i$).
    On the other hand, $L(f, P)$ will always be $0$.
    For any $\epsilon > 0$, take $P_\epsilon$ to be a partition such that
    $U(f, P_\epsilon) < \epsilon$, then $U(f, P_\epsilon) - L(f, P_\epsilon) < \epsilon$.
    Therefore, $\displaystyle \int\limits_0^1 f(x) \d x = 0$.
  \end{proof}
  % \end{answer}


\newpage
\begin{problem}
  Prove that if the real-valued function $f$ on the interval $[a, b]$
  is integrable on $[a, b]$, then so is $\abs{f}$, and
  \[
    \abs{\int\limits_a^b f(x) \d x} \leq \int\limits_a^b \abs{f(x)} \d x.
  \]
\end{problem}

% \begin{answer}
  \begin{claim}
    Suppose $f$ is integrable on $[a, b]$, then so is $\abs{f}$.
  \end{claim}
  \begin{proof}
    Since $f$ is integrable on $[a, b]$, for any $\epsilon > 0$,
    there exists a partition $P = \set{x_0, x_1, \ldots, x_n}$ of $[a, b]$
    such that
    \[ U(f, P) - L(f, P) < \epsilon. \]
    Since \blue{$\abs{a} - \abs{b} \leq \abs{a - b}$}\footnote{
      This can be shown using Cauchy-Schwarz inequality:
      \begin{alignat*}{5}
        &\abs{a} = \abs{(a - b) + b} &&\leq \abs{a - b} + \abs{b}
          &&\qquad \text{(triangle inequality)}\\
        &\abs{a} - \abs{b} &&\leq \abs{a - b} &&\qquad \text{(deduct $\abs{b}$ to both sides)}
      \end{alignat*}
    }
    for all $a, b \in \R$, we have
    $U(\abs{f}, P) - L(\abs{f}, P) \leq U(f, P) - L(f, P) < \epsilon$,
    so $\abs{f}$ is also integrable on $[a, b]$.
  \end{proof}

  \begin{claim}
    Given that both $f$ and $\abs{f}$ are integrable on $[a, b]$,
    then $\abs{\int\limits_a^b f(x) \d x} \leq \int\limits_a^b \abs{f(x)} \d x$.
  \end{claim}
  \begin{proof}
    Since $f$ is integrable on $[a, b]$, for any $\epsilon > 0$,
    there exists a partition $P = \set{x_0, x_1, \ldots, x_n}$ of $[a, b]$
    such that
    \[ U(f, P) - L(f, P) < \epsilon, \]
    and \[ L(f, P) \leq R(f, P) \leq U(f, P), \]
    with $\displaystyle R(f, P) = \sum\limits_1^n f(c_i)$ for some
    $c_i \in [a_{i-1}, a_i]$.
    As $\norm{P} \to 0$, $\epsilon \to 0$, so the Riemann sum $R(f, P)$
    converges to the integral;
    \[
      \blue{
        \lim_{\norm{P} \to 0} \sum_{i=1}^n f(c_i) \Delta x_i =
        \int\limits_a^b f(x) \d x}
      \qquad \text{ and } \qquad
      \blue{
        \lim_{\norm{P} \to 0} \sum_{i=1}^n \abs{f(c_i)} \Delta x_i =
        \int\limits_a^b \abs{f(x)} \d x}.
    \]
    Thus, our original claim is equivalent to:
    \begin{alignat*}{5}
      &\abs{\int\limits_a^b f(x) \d x} &&= \abs{\lim_{\norm{P} \to 0}
        \sum_{i=1}^n f(c_i) \Delta x_i}
        &&\qquad \blue{\text{(for some $c_i \in [x_{i-1}, x_i]$)}}\\
      & &&\qquad \leq \lim_{\norm{P} \to 0} \sum_{i=1}^n \abs{f(c_i)} \Delta x_i
        &&\qquad \blue{\text{(triangle inequality)}} \\
      & &&\qquad \qquad= \int\limits_a^b \abs{f(x)} \d x,
    \end{alignat*}
  \end{proof}
  % reduce bottom margin
  \vspace{-3em}
% \end{answer}


\newpage
\begin{problem}
  Prove integration by parts.
  That is, suppose $F$ and $G$ are continuously differentiable functions on $[a, b]$.
  Then, prove that
  \[
    \int\limits_a^b F(x)G'(x) \d x = F(b)G(b) - F(a)G(a) - \int\limits_a^b F'(x)G(x) \d x.
  \]
\end{problem}

% \begin{answer}
  \begin{proof}
    Since $F$ and $G$ are continuously differentiable on $[a, b]$,
    they are also continuous on $[a, b]$.
    Thus, by the fundamental theorem of calculus, we have
    Let $H(x) = F(x)G(x)$, then
    \begin{align*}
      H'(x) &= \frac{d}{dx} \brackets{F(x)G(x)} \\
            &= F'(x)G(x) + F(x)G'(x) \qquad \blue{\text{(by the product rule)}}. \\
      \intertext{ What happens if we integrate both sides of this equation? }
      \int\limits_a^b H'(x) \d x &= \int\limits_a^b \brackets{F'(x)G(x) + F(x)G'(x)} \d x \\
      H(b) - H(a) &= \int\limits_a^b F'(x)G(x) \d x + \int\limits_a^b F(x)G'(x) \d x
    \end{align*}
    Rearranging this gives us
    \[
      H(b) - H(a) - \int\limits_a^b F'(x)G(x) \d x = \int\limits_a^b F(x)G'(x) \d x
    \]

    Since $H(x) = F(x)G(x)$, we have
    \blue{\[
      \int\limits_a^b F(x)G'(x) \d x = F(b)G(b) - F(a)G(a) - \int\limits_a^b F'(x)G(x) \d x
    \]}
  \end{proof}
% \end{answer}


\newpage
\begin{problem}
  Let $g, f : \R \to \R$ be Riemann integrable on any interval $[a, b] \subset \R$.
  Is it true that $g \circ f$ is also Riemann integrable on any interval $[a, b] \subset \R$?\\
  \underline{Hint:} Consider $g$ such that $g(x) = 0$ if $x = 0$
  and $g(x) = 1$ if $x \neq 0$, and $f$ as in Problem \crim{$3$}:
  \[
    f(x) = \begin{cases}
      0 & \text{ if } x \in \R \setminus \Q \text{ (i.e. not rational).}\\
      \frac{1}{q} & \text{ if } x \colonequals \frac{p}{q} \in \Q
      \text{ with $p, q$ coprime and $q > 0$}.
    \end{cases}
  \]
\end{problem}

% \begin{answer}
  \begin{claim}
    The composition of two Riemann integrable functions is not necessarily Riemann integrable.
  \end{claim}
  \begin{proof}
    Consider the functions $g, f : \R \to \R$ as defined anove,
  and their composition $g \circ f$.
  Both $g$ and $f$ are Riemann integrable on any interval $[a, b] \subset \R$,
  as priorly shown. However, Let's look at $g \circ f$:

  \[
    (g \circ f)(x) = \begin{cases}
      0 & \text{ if } x \in \R \setminus \Q \text{ (i.e. not rational).}\\
      1 & \text{ if } x \colonequals \frac{p}{q} \in \Q
      \text{ with $p, q$ coprime and $q > 0$}.
    \end{cases}
  \]


  Note that $g \circ f$ is not continuous at any point $x \in \Q$,
  since there exists a rational number between any two distinct irrationals
  \footnote{
    \textbf{Proof that there exists a rational number between any two distinct irrationals.} \\
    Let $a, b \in \R \setminus \Q$ with $a < b$.
    Let $c = b - a$. By the properties of $\R$, there exists $n \in \N$
    such that $n > 1/c$, which implies that $cn > 1$.
    Since we took $c = b - a$, this implies that $nb - na > 1$.
    Therefore, there exists some integer $N$ such that $na < N < nb$.
    Dividing by $n$, we get $a < N/n < b$.
    Thus, $N/n$ is a rational number between $a$ and $b$. \\
  }
  and there exists an irrational number between any two distinct rationals
  \footnote {
    \textbf{Proof that there exists an irrational number between any two distinct rationals.} \\
    Let $a, b \in \Q$ with $a < b$.
    Then $b - a > 0$, $\displaystyle b - a > \frac{b-1}{\sqrt{2}}$,
    and $\displaystyle \frac{b - a}{\sqrt{2}} \not \in \Q$.
    Therefore $\displaystyle a + \frac{b - a}{\sqrt{2}} \in \R \setminus \Q$, and it is contained
    in the interval $(a, b)$. \\
  }.
  Consequently, if we take any $0 < \epsilon < 1$, then there is no value
  for $\delta > 0$ that satisfies the $\epsilon$-$\delta$ criterion for continuity
  at any point $x \in \R$ since. Take $x_2$ to be any number in the interval
  $(x - \delta, x + \delta)$, then:
  \begin{enumarabic}
    \item If $x \in \Q$ and $x_2 \in \R \setminus \Q$, then
      $(g \circ f)(x) = 1$ and $(g \circ f)(x_2) = 0$,
      so \[ \abs{(g \circ f)(x) - (g \circ f)(x_2)} = 1 > \epsilon. \]
    \item If $x \in \Q$ and $x_2 \in \Q$, then there exists some irrational number
      $x_3$ between $x$ and $x_2$, then $(g \circ f)(x) = 1$
      and $(g \circ f)(x_3) = 0$. Therefore,
      \[ \abs{(g \circ f)(x) - (g \circ f)(x_3)} = 1 > \epsilon. \]
    \item If $x \in \R \setminus \Q$ and $x_2 \in \Q$, then
      $(g \circ f)(x) = 0$ and $(g \circ f)(x_2) = 1$,
      so \[ \abs{(g \circ f)(x) - (g \circ f)(x_2)} = 1 > \epsilon. \]
    \item If $x \in \Q$ and $x_2 \in \Q$, then there exists some irrational number
      $x_3$ between $x$ and $x_2$, then $(g \circ f)(x) = 1$
      and $(g \circ f)(x_3) = 0$. Therefore,
      \[ \abs{(g \circ f)(x) - (g \circ f)(x_3)} = 1 > \epsilon. \]
  \end{enumarabic}

  \step
  Thus,  $g \circ f$ is not continuous at any point $x \in \R$,
  therefore not Riemann integrable on any interval $[a, b] \subset \R$.
\end{proof}
 
% \end{answer}


\vfill

\end{document}
