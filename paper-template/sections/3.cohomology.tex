\section{Hochschild cohomology}
\label{sect:cohomology}

In this section we present a rather effortless calculation of the
Hochschild cohomology of the algebra~$A$. Our approach is directly targeted
at obtaining completely explicit representatives of cohomology classes.
As before, we suppose throughout that $N\geq1$. The Hochschild cohomology
of the first Weyl algebra, the algebra we get when $N=0$, is
well-know to be the same as that of the ground field --- this was computed
originally by Ramaiyengar Sridharan in~\cite{Sridharan}.

\bigskip

Let $V$ be the subspace of~$A$ spanned by~$x$ and~$y$, let $V^*$ be its
dual vector space, and let $(\hat x,\hat y)$ be the ordered basis of~$V ^*$
dual to~$(x,y)$. There is a projective resolution~$P_*$ of~$A$ as an $A$-bimodule
of the form
  \[ \label{eq:res}
  \begin{tikzcd}
  0 \arrow[r]
    & A\otimes\Lambda^2V\otimes A\arrow[r, "d_2"]
    & A\otimes V\otimes A\arrow[r, "d_1"]
    & A\otimes A
  \end{tikzcd}
  \]
with differentials such that
  \begin{align}
  d_1(1\otimes x\otimes 1) 
        &= x\otimes 1-1\otimes x, \\
  d_1(1\otimes y\otimes 1) 
        &= y\otimes 1-1\otimes y, \\
  d_2(1\otimes x\wedge y\otimes 1) 
        &= 
         \begin{multlined}[t][0.6\displaywidth]
          y\otimes x\otimes 1 + 1\otimes y\otimes x
          - x\otimes y\otimes 1 - 1\otimes x\otimes y \\
          - \sum_{s+1+t=N}x^s\otimes x\otimes x^t
         \end{multlined}
  \end{align}
and augmentation $\epsilon:A\otimes A\to A$ given by the multiplication
of~$A$. Clearly, $V$ is a homogeneous subspace of~$A$, and if we endow it
with the induced grading, and in turn $\Lambda^2V$ with the grading induced
by that of~$V$ and each component of the complex~\eqref{eq:res} with the
obvious tensor product gradings, that complex becomes a complex of graded
$A$-bimodules. Applying to it the functor~$\hom_{A^e}(-,A)$ we obtain, up
to standard identifications, the cochain complex
  \[ \label{eq:comp}
  \begin{tikzcd}
    & A \arrow[r, "\delta_0"]
    & A\otimes V^* \arrow[r, "\delta_1"]
    & A\otimes \Lambda^2V^* \arrow[r]
    & 0
  \end{tikzcd}
  \]
with differentials such that
  \begin{align}
  \delta_0(a) &= [x,a]\otimes\hat x+[y,a]\otimes\hat y, \\
  \delta_1(b\otimes\hat x+c\otimes\hat y) &=
    \left([y,b] + [c,x] - \sum_{s+1+t=N}x^sbx^t\right)
    \otimes\hat x\wedge\hat y
  \end{align}
for all choices of~$a$, $b$ and~$c$ in~$A$. The cohomology of this complex
is canonically isomorphic to the Hochschild cohomology~$\HH^\bullet(A)$
of~$A$, and we \emph{identify} the two. If we grade~$V^*$ so that $\hat x$
and~$\hat y$ have degrees~$-1$ and~$-(N-1)$, respectively, then the
complex~\eqref{eq:comp} is one of graded vector spaces and homogeneous maps
of degree~$0$, and, consequently, its cohomology~$\HH^\bullet(A)$ is also
graded. This grading on Hochschild cohomology coincides with the canonical
grading it gets from the fact that the algebra~$A$ is graded and has a free
resolution as a graded bimodule over itself by finitely generated free
bimodules --- we will not belabor this point, but it is important that the
grading we found is \emph{the} grading on~$\HH^\bullet(A)$.

\bigskip

The hardest part of the calculation is that of~$\HH^1(A)$, and we will
leave it for the end, as we will do it in a rather indirect way. As for
that of~$\HH^0(A)$, we have actually already done it:

\begin{Lemma}
The $0$th cohomology space $\HH^0(A)$, the kernel of the map~$\delta_0$, is
$\kk$, and its Hilbert series is therefore $h_{\HH^0(A)}(t) = 1$.
\end{Lemma}

\begin{proof}
We established in Proposition~\ref{prop:center} that the center of~$A$,
which coincides with the kernel of the map~$\delta_0$, is~$\kk$ and then
that its Hilbert series is~$1$ is clear. 
\end{proof}

We know from Lemma~\ref{lemma:normal} that $x^N$ is a normal element
in~$A$, so that the right ideal~$x^NA$ is a bilateral ideal. Moreover, that
ideal is related to commutators in the following way:

\begin{Lemma}\label{lemma:xN}
We have $[A,x]=[A,A]=x^NA$.
\end{Lemma}

\begin{proof}
If $i$,~$j\geq0$, then 
  \[
  [x^iy^j,x] 
        = \sum_{s+1+t=j}x^iy^s[y,x]y^t
        = \sum_{s+1+t=j}x^iy^sx^Ny^t
        \in x^NA
  \]
because~$x^NA$ is an ideal, and this implies that $[A,x]\subseteq x^NA$.

To prove the reverse inclusion, we will show that $x^{i+N}y^j\in[A,x]$ for
all $i$,~$j\geq0$ by induction on~$j$. If $j\geq0$ and
$x^{i+N}y^k\in[A,x]$ for all $i\geq0$ and all $k\in\{0,\dots,j-1\}$ ---~this
hypothesis is vacuous when $j=0$, and this starts the induction~--- then
there is an $u\in F_{j-1}$ such that $[x^iy^{j+1},x] = (j+1)x^{i+N}y^j +
u$ and, because of the hypothesis, a $v\in A$ such that $u=[v,x]$: we then
have that $x^{i+N}y^j=[(j+1)^{-1}x^iy^{j+1}-v,x]\in[A,x]$. The induction is
thus complete.

If $i$,~$j$,~$k$,~$l\geq0$, then we have
  \[
  [x^iy^j,x^ky^l]
        = x^i[y^j,x^k]y^l + x^k[x^i,y^l]y^j,
  \]
so to prove that $[A,A]$ is contained in~$x^NA$, and thus the rest of the
equalities asserted by the lemma, it is enough to compute
that
  \[
  [y^j,x^k]
        = \sum_{\substack{s+1+t=j\\u+1+v=k}}
                y^sx^u[y,x]x^vy^t
        \in x^NA
  \]
because~$[y,x]=x^N$ and $x^NA$ is a bilateral ideal.
\end{proof}

The description of~$[A,A]$ that we have now allows us to compute~$\HH^2(A)$
easily.

\begin{Lemma}
The image of the map~$\delta_1$ is $x^{N-1}A\otimes\hat x\wedge\hat y$, and
therefore 
  \[
  \HH^2(A)\cong A/x^{N-1}A\otimes\hat x\wedge\hat y.
  \]
If $N=1$ then this is of course~$0$, and if $N\geq2$, so that $\HH^2(A)$ is
graded with finite-dimensional homogeneous components, then the
Hilbert series if this space is
  \[
  h_{\HH^2(A)}(t) = \frac{t^{-N}}{1-t}.
  \]
\end{Lemma}

\begin{proof}\allowdisplaybreaks
For all $c\in A$ we have $\delta_1(c\otimes\hat y)=[c,x]\otimes\hat
x\wedge\hat y$, and this, together with Lemma~\ref{lemma:xN}, tells us that
$x^NA\otimes\hat x\wedge\hat y=\delta_1(A\otimes\hat
y)\subseteq\img\delta_1$. On the other hand, if $i$,~$j\geq0$, we have that
  \begin{align}
  \delta_1(x^iy^j\otimes\hat x)
        &= \left([y,x^iy^j] - \sum_{s+1+t=N}x^{s+i}y^jx^t\right)
           \otimes\hat x\wedge\hat y \\
        &= \left(ix^{i+N-1}y^j - \sum_{s+1+t=N}x^{s+i}y^jx^t\right)
           \otimes\hat x\wedge\hat y \\
        &= \left((i-N)x^{i+N-1}y^j - \sum_{s+1+t=N}x^{s+i}[y^j,x^t] \right)
           \otimes\hat x\wedge\hat y. 
  \end{align}
According to Lemma~\ref{lemma:xN}, the sum appearing in this last
expression is an element of~$x^NA$: it follows from this that
$\delta_1(x^iy^j\otimes\hat x)\subseteq x^{N-1}A$, and then that
  \[
  x^NA 
        \subseteq \img\delta_1 
        = \delta_1(A\otimes\hat x)+\delta_1(A\otimes\hat y)
        \subseteq x^{N-1}A + x^{N}A = x^{N-1}A.
  \]
Moreover, for each $j\geq0$ we have that
  \[
  \delta_1(y^j\otimes\hat x)
        = \left(-Nx^{N-1}y^j - \sum_{s+1+t=N}x^s[y^j,x^y] \right)
           \otimes\hat x\wedge\hat y.
  \]
Since the sum appearing here is in~$x^NA$ it is equal to~$[b,x]$ for
some~$b\in A$ and therefore
  \[
  -Nx^{N-1}y^j\otimes\hat x\wedge\hat y = \delta_1(y^j\otimes\hat x+b\otimes\hat y) 
        \in \img\delta_1.
  \]
Putting everything together we conclude that $\img\delta_1=x^{N-1}A$, as
we want. Finally, the quotient $A/x^{N-1}A$ is freely spanned by the
classes of the monomials~$x^iy^j$ with $0\leq i<N-1$ and $j\geq0$, and
there is exactly one such monomial of each degree in~$\NN_0$: the Hilbert
series of~$A/x^{N-1}A$ is thus $(1-t)^{-1}$, and the Hilbert series
of~$\HH^2(A)$ is as described in the lemma, because of the factor~$\hat
x\wedge\hat y$.
\end{proof}

At this point we know the Hilbert series of~$\HH^0(A)$ and of~$\HH^2(A)$,
and we can use the invariance of the Euler characteristic of a complex
under the operation of taking cohomology to determine the Hilbert series
of~$\HH^1(A)$.

\begin{Proposition}\label{prop:hh1-series}
If $N\geq2$, then
the Hilbert series of~$\HH^1(A)$ is
  \[
  h_{\HH^1(A)}(t) = 1 + \frac{t^{-N+1}}{1-t},
  \]
so that for all integers~$l$ we have that
  \[
  \dim\HH^1(A)_l =
    \begin{cases*}
    2 & if $l=0$; \\
    1 & if $l\geq-N+1$ and $l\neq0$; \\
    0 & if $l\leq -N$.
    \end{cases*}
  \]
\end{Proposition}

\begin{proof}
The Hilbert series
of~$A$ is
  \[
  h_A(t) = \frac{1}{(1-t)(1-t^{N-1})},
  \]
and then the Euler characteristic of the complex~\eqref{eq:comp} is
  \[
  h_A(t) - (t^{-1}+t^{-(N-1)})h_A(t)+t^{-N}h_A(t) = t^{-N}.
  \]
The invariance of the Euler characteristic when passing to cohomology
implies now that
  \[
  t^{-N} = h_{\HH^0(A)}(t) - h_{\HH^1(A)}(t) + h_{\HH^2(A)}(t),
  \]
and one can compute~$h_{\HH^1(A)}(t)$ from this equality, since we know the
values of both $h_{\HH^0(A)}(t)$ and~$h_{\HH^2(A)}(t)$, finding the formula
given in the proposition.
\end{proof}

In Proposition~\ref{prop:hh1-series} we excluded the case in which $N=1$,
which is special --- for one thing, the graded algebra~$A_N$ is not locally
finite-dimensional in that case, so we cannot even talk about its Hilbert
series! Let us deal with it now for the sake of completeness.

\begin{Proposition}\label{prop:hh1-n1}
If $N=1$, then there is a derivation~$d_0:A\to A$ such that $d_0(x)=0$
and~$d_0(y)=1$. It is homogeneous of degree~$0$, and its cohomology class
freely spans the vector space~$\HH^1(A)$, which is thus $1$-dimensional and
concentrated in degree~$0$.
\end{Proposition}

When $N=1$ the map $\ad(y):u\in A\mapsto [y,u]\in A$ acts on elements of
each degree~$l\in\ZZ$ by multiplication by~$l$. We will use this fact in
the proof.

\begin{proof}
A trivial calculation shows that there is indeed a derivation~$d_0:A\to A$
with $d_0(x)=0$ and $d_0(y)=1$, and it is clearly homogeneous of
degree~$0$. Were it inner, we would have an element~$c$ such that $[x,c]=0$
and $[y,c]=1$: the first equality implies that $c\in\kk[x]$, and then the
second one that $x^Nc'=1$, which is absurd.

Let now $d:A\to A$ be an arbitrary homogeneous derivation and let $l$ be
its degree. If~$a\coloneqq d(x)$ and~$b\coloneqq d(y)$, then $a$ and~$b$
are homogeneous of~$A$ of degree~$l+1$ and~$l$, respectively. As $d$ is a
derivation, we have that
  \[ \label{eq:zero}
  0 = d([y,x]-x) = [b,x] + [y,a] - a = [b, x] + la.
  \]
If $l\neq0$ this tells us that $a=-\frac{1}{l}[b,x]$, and using this we see
at once that $d$ is the inner derivation~$\tfrac{1}{l}\ad(b)$. Let us then
suppose that $l=0$. Now the equality above tells us that $b$
commutes with~$x$, so that is an homogeneous element of~$\kk[x]$ of
degree~$0$: in other words, we have $b\in\kk$. On the other hand, since $a$
has degree~$1$, there is an $f\in\kk[y]$ such that $a=xf$. There exists a
$g\in\kk[y]$ such that $f=g(y+1)-g(y)$, and then $[g,x]=a$. The
derivation~$d'\coloneqq d-\ad(g)$ is therefore homogeneous of degree~$0$,
cohomologous to~$d$, and has $d'(x)=0$ and $d'(y)=b$. We can thus conclude
that every homogeneous derivation is cohomologous lo a multiple of the
derivation~$d_0$. This proves the proposition.
\end{proof}

If we are asked for a reason explaining the difference between the case in
which $N=1$ and that in which $N\geq2$ exhibited by the last two results, a
good candidate for an answer is the following. Whatever the value of~$N$,
the algebra~$A$ has a grading, $A=\bigoplus_{j\geq0}A_j$, and there is
therefore a derivation $E:A\to A$ such that for all $j\geq0$ the
component~$A_j$ is invariant under~$E$ and the restriction~$E|_{A_j}:A_j\to
A_j$ is simply multiplication by the scalar~$j$. This is a diagonalizable
derivation and the homogeneous components of~$A$ are precisely its
eigenspaces. Now a difference arises: if $N=1$, the derivation~$E$ is
inner, since it coincides with~$\ad(y)$, while if~$N\geq2$ the derivation
not inner. This is behind the collapse of~$\HH^1(A)$ that occurs when
$N=1$.
