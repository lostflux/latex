\phantomsection
\addcontentsline{toc}{section}{Introduction}%
\section*{Introduction}

\begin{figure}[H]
  \centering
  \begin{tikzpicture}
    \node[state, initial, accepting] (q0) at (0, 0) {$q_0$};
    \node[state, right of=q0] (q1) {$q_1$};
    \draw (q0) edge[above] node {$0, 1$} (q1)
          (q1) edge[loop above, above] node {$0, 1$} (q1);
  \end{tikzpicture}
  \caption{Only accept the empty string.}
  \label{fig:empty}
\end{figure}

In this paper we fix a field~$\kk$ of characteristic zero and a
non-negative integer~$N$, and study the algebra~$A_N$ freely
generated by two letters~$x$ and~$y$ subject to the relation
  \[
  yx-xy = x^N.
  \]
with the objective of computing as explicitly as it is possible (to us!)
some of its invariants of homological nature.

For low values of $N$ the algebra $A_N$ is very well-known: when $N=0$ it is
the first Weyl algebra, which we can view as the algebra of regular
differential operators on the affine line; when $N=1$ it is the enveloping
algebra of the non-abelian Lie algebra of dimension~$2$; and when $N=2$ it
is the so-called Jordan plane of non-commutative geometry
\citelist{\cite{AS} \cite{SZ}}. On the other hand, the family of algebras
that we will study is contained in a larger one that has received a lot of
attention: if for an arbitrary polynomial $h\in\kk[x]$ we let $A_h$ be the
algebra freely generated by letters~$x$ and~$y$ subject to the relation
$yx-xy=h(x)$, then of course our algebra~$A_N$ is $A_{x^N}$. One way to
explain the interest of this larger family of algebras is by saying that it
consists, up to isomorphism, of all skew-polynomial extensions of~$\kk[x]$
apart from the $1$-parameter families of quantum planes and quantum Weyl algebras
\cite{AD}*{Proposition 3.2}. The algebras of the form~$A_h$ have been
studied in detail by G.~Benkart, S.~Lopez and M.~Ondrus in the series of
papers \citelist{\cite{BLO:1} \cite{BLO:2} \cite{BLO:3}}. Our
algebras~$A_N$ are, in many senses, the \emph{worst} of the lot.

The main motivation for this work was the problem of giving an explicit
description of the first Hochschild cohomology space $\HH^1(A_N)$ of~$A_N$,
which we view as the space of outer derivations of the algebra~$A_N$, that
is, the quotient $\Der(A_N)/\InnDer(A_N)$ of the Lie algebra $\Der(A_N)$ of
all derivations of~$A_N$ by its ideal of inner derivations. This cohomology
space and, in fact, the full Lie algebra $\Der(A_N)$ have been studied in
detail before --- J.~Dixmier \citelist{\cite{Dixmier:1} \cite{Dixmier:2}}
and R.~Sridharan \cite{Sridharan} for the Weyl algebra, E.N.~Shirikov
\citelist{\cite{Shirikov:1} \cite{Shirikov:2} \cite{Shirikov:3}} for the
case $N=2$, A.~Nowicki \cite{Nowicki} and most notably G.~Benkart, S.~Lopez
and M.~Ondrus \cite{BLO:3} for the general case of the algebras~$A_h$ and,
building upon that, S.~Lopez and A.~Solotar \cite{LSo} for the Lie module
structure of the cohomology of~$A_h$ over~$\HH^1(A_h)$ --- and we can say
that both $\Der(A_N)$ and $\HH^1(A_N)$ are very well understood. What we
were after, though, was a description of the elements of~$\HH^1(A_N)$ as
cohomology classes of actual, explicit derivations, because we needed to do
further calculations with them. In particular, while doing some
calculations regarding the characteristic morphism
$\HH^\bullet(A_N)\to\mathscr{Z}_{\gr}(D^b(A_N))$ for this algebra --- which
connects the Hochschild cohomology of the algebra with the graded center of
the derived category of the category of modules of the algebra, for example
as in \cite{Lowen} --- certain rational numbers insistently appeared and
required an explanation.

\bigskip

Let us describe the results we obtain. The algebra~$A_N$ can be endowed
with a grading with respect to which the generators~$x$ and~$y$
are in degrees~$1$ and~$N-1$, respectively, and this grading induces others
in many objects constructed from~$A_N$. For example, the space~$\Der(A_N)$
and, for each $p\geq0$, the Hochschild cohomology space $\HH^p(A_N)$ are
$\ZZ$-graded vector spaces. Our result about~$\HH^1(A_N)$ is the following:

\begin{theorem}\label{thm:hh1}
Suppose that $N\geq2$, let $q$ be a variable, and let $(c_i(q))_{i\geq0}$
be the sequence of polynomials in~$\QQ[q]$ such that
  \[ \label{eq:cj}
  \sum_{j\geq0}c_j(q)\frac{t^j}{j!} = \frac{t}{(1-qt)^{-1/q}-1}.
  \]
\begin{thmlist}

\item If $l$ is a positive integer and $i$ and~$j$ are the integers such
that $l+1=i+j(N-1)$ and $1\leq i\leq N-1$, then there is a homogeneous
derivation $\partial_l:A_N\to A_N$ of degree~$l$ such that
  \begin{align}
  \partial_l(x) &= x^iy^j, \\
  \partial_l(y) &= 
        \begin{multlined}[t][.75\displaywidth]
        (N-i)x^{i-1}
        \frac{1}{j+1}
        \sum_{i=0}^{j}
        \binom{j+1}{i}
        c_i(N-1)
        x^{i(N-1)}y^{j+1-i} \\
        +\sum_{s+2+t=N}(s+1)x^{s+i}y^jx^t 
        \end{multlined}
        \label{eq:ply}
  \end{align}

\item If $l$ is an integer such that $-N+1\leq i\leq 0$, then there is a
unique homogeneous derivation $\partial_l:A_N\to A_N$ of degree~$l$ such that
  \[
  \partial_l(x) = 0, 
  \qquad 
  \partial_l(y) = x^{l+N-1}.
  \]
There is moreover a homogeneous derivation~$E:A_N\to A_N$ of degree~$0$
such that
  \[
  E(x) = x,
  \qquad
  E(y) = (N-1)y.
  \]

\item The graded vector space~$\HH^1(A_N)$ is locally finite and its Hilbert
series is 
  \[
  h_{\HH^1(A_N)}(t) = 1 + \frac{t^{-N+1}}{1-t}.
  \]
If $l$ is an non-zero integer such that $l\geq-N+1$, then the homogeneous
component~$\HH^1(A_N)_l$ of degree~$l$ in~$\HH^1(A_N)$ is freely spanned by
the cohomology class of the derivation $\partial_l$ described above. On the
other hand, the component $\HH^1(A_N)_0$ of degree~$0$ is freely spanned by
the cohomology classes of the derivations~$\partial_0$ and~$E$.

\end{thmlist}
\end{theorem}

The sequence of polynomials $(c_j(q))_{q\geq0}$ that appears here --- of
which the first few are tabulated in Table~\vref{tbl:ciq} --- is a
$q$-variant of the sequence of Bernoulli numbers, to which it degenerates
as $q$ goes to~$0$: indeed, the limit of the right hand side of the
defining equality~\eqref{eq:cj} as~$q$ approaches~$0$ is $t/(e^t-1)$, the
exponential generating function of the Bernoulli numbers, and 
for all $j\in\NN_0$ the constant term $c_j(0)$ is exactly the $j$th
Bernoulli number. On the other hand, the leading coefficient of~$c_j(q)$ is
$(-1)^jj!G_j$, with $G_j$ denoting the $j$th Gregory coefficient, certainly of much
lesser fame.

It should be observed that, while the formula in~\eqref{eq:ply} somehow
indicates that the limiting case $q\leadsto0$ corresponds to letting the
integer~$N$ «converge» to~$1$ (which, of course, makes no sense), none of
the claims of the theorem holds at the limit $N=1$! Indeed, the theorem
excludes the cases in which $N<2$, and that is because they are rather
different. When $N=0$ we have, according to a calculation carried out
originally by Dixmier in~\cite{Dixmier:2}, that $\HH^1(A)=0$, so there is
nothing that needs being made explicit. When $N=1$, on the other hand, the space
$\HH^1(A)$ is one-dimensional and spanned by the cohomology class of a
derivation $d_0:A\to A$ such that $d_0(x)=0$ and $d_0(y)=1$, which is
homogeneous of degree~$0$.

The way in which we prove this theorem is rather indirect --- a direct
proof of the first part of the statement would probably be quite
unpleasant! We first compute $\HH^0(A)$ and~$\HH^2(A)$ and, in particular,
their Hilbert series, and using that and an argument involving Euler
characteristics we deduce the Hilbert series of~$\HH^1(A)$: this tells us
of what degrees there are non-inner homogeneous derivations, and how many.
We then show that such a derivation can be modified appropriately until it
satisfies the conditions described in the theorem. We do this in
Sections~\ref{sect:cohomology}, \ref{sect:phi} and~\ref{sect:explicit},
whose main results are the Propositions~\ref{prop:hh1-series},
\ref{prop:hh1-n1}, \ref{prop:hh1-low} and~\ref{prop:hh1-high} that we
subsumed in Theorem~\ref{thm:hh1} above.

\bigskip

Once we have explicit derivations whose cohomology classes
span~$\HH^1(A_N)$, we can do things with them. The general qualitative
structure of the Lie algebra~$\HH^1(A)$ was described by G.~Benkart,
S.~Lopez and M.~Ondrus in \cite{BLO:3}: its center is one-dimensional and a
complement to the derived subalgebra
$\HH^1(A_N)'\coloneqq[\HH^1(A_N),\HH^1(A_N)]$, this derived subalgebra has
a unique maximal nilpotent ideal~$\Nil$ of nilpotency index~$N$, and the
quotient $\HH^1(A_N)'/\Nil$ is isomorphic to the Lie algebra $\Der(\kk[x])$
of derivations of~$\kk[x]$ or, equivalently, of regular vector fields on
the affine line~$\AA^1_\kk$, which is often called the Witt algebra.
Starting from Theorem~\ref{thm:hh1} we can prove --- this is
Corollary~\ref{coro:mlmm} in the text below --- the following:

\begin{Theorem}
There is a sequence of derivations $(L_j)_{j\geq-N+1}$ of~$A_N$ whose
cohomology classes freely span the derived subalgebra~$\HH^1(A)'$ such that
whenever $l$ and~$m$ are integers with $l$,~$m\geq-N+1$ we have
  \[ 
  [L_l, L_m] \sim
    \begin{dcases*}
    0 & if $i+u>N$ or $l+m<-N+1$; \\
    \frac{l(v+1)-m(j+1)}{N-1}L_{l+m} & if $i+u\leq N$,
    \end{dcases*}
  \]
with $i$, $j$, $u$ and~$v$ the unique integers such that
$l+1=i+j(N-1)$, $m+1=u+v(N-1)$, $1\leq i, u\leq N-1$, and
$j$,~$v\geq-1$. 
\end{Theorem}

In fact, for each non-zero $j$ the derivation~$L_j$ here is just a scalar
multiple of the derivation~$\partial_j$ of Theorem~\ref{coro:mlmm}, $L_0$
is a scalar multiple of the derivation~$E$, and the center of~$\HH^1(A)$ is
the span of the class of~$\partial_0$. In particular, the degree of~$L_j$
is~$j$.

This shows that the derived subalgebra~$\HH^1(A)'$ is, more or less, an
infinitesimal deformation of order~$N$ of the Witt algebra~$\Der(\kk[x])$
--- which is now realized as the subalgebra spanned by the sequence
$(L_{(-N+1)j})_{j\geq-1}$. It would be very interesting to have \emph{a
priori} reasons for this.

\bigskip

The first Hochschild cohomology of an algebra plays ---~in principle, but
usually not in reality ---  the role of the Lie algebra of the group of
outer automorphism of the algebra. For our algebra this does not quite
work, and since the units of~$A_N$ are all scalar and therefore central its
usual outer automorphism group coincides with the plain automorphism group.
There is an alternative notion of inner-automorphism that is useful in Lie
theory, though, in which we call an automorphism inner if it is a
composition of exponentials of locally $\ad$-nilpotent elements, and in our
situation it does do something, as the following result shows.

\begin{Theorem}
Let $\kk[x]\bowtie\kk^\times$ be the group whose underlying set is the
cartesian product $\kk[x]\times\kk^\times$ and whose multiplication is such that
  \[
  (f,\lambda)\cdot(g,\mu) = (\mu^{N-1}f+g\cdot\lambda,\lambda\mu)
  \]
for all $f$,~$g\in\kk[x]$ and all~$\lambda$,~$\mu\in\kk^\times$. 
\begin{thmlist}

\item There is
an isomorphism of groups
  \[
  \Phi:\kk[x]\bowtie\kk^\times\to\Aut(A_N)
  \]
such that $\Phi(f,\lambda)(x) = \lambda x$ and
$\Phi(f,\lambda)(y) = \lambda^{N-1}y + f$
for all $(f,\lambda)$ in~$\kk[x]\bowtie\kk^\times$.

\item The set of locally $\ad$-nilpotent elements of~$A_N$ is $\kk[x]$, and
for each $f\in\kk[x]$ we have $\exp\ad(f)=\phi_{x^Nf,1}$. The subset
$\Exp(A_N)$ of exponentials of locally $\ad$-nilpotent elements is a normal
subgroup of the automorphism group~$\Aut(A_N)$, and the map~$\Phi$ above
induces an isomorphism
  \[
  \overline\Phi:
  \frac{\kk[x]}{(x^N)}\bowtie\kk^\times
  \to
  \frac{\Aut(A_N)}{\Exp(A_N)}.
  \]
In particular, the quotient $\Aut(A_N)/\Exp(A_N)$ has a natural structure
of a Lie group over~$\kk$ of dimension~$N+1$, solvable of class~$2$ and, in
fact, an extension of~$\kk^\times$ by~$\kk^N$.

\item For each $g\in\kk[x]$ there is a derivation~$d_g:A_N\to A_N$ such
that $d_g(x)=0$ and $d_g(y)=g$, and it is locally nilpotent. The map
  \[
  g\in\kk[x]\mapsto d_g\in\Der(A_N)
  \]
is injective, and its image is the set of locally nilpotent derivations
of~$A_N$, which happens to be an abelian Lie subalgebra of~$\Der(A_N)$. The
set of exponentials of locally nilpotent derivations of~$A_N$ 
is the normal subgroup
  \(
  \Aut_0(A_N) \coloneqq \{\phi_{g,1}:g\in\kk[x]\}
  \),
and it sits in an extension of groups
  \[
  \begin{tikzcd}
  0 \arrow[r]
    & \Aut_0(A) \arrow[hook, r]
    & \Aut(A) \arrow[r, "\det"]
    & \kk^\times \arrow[r]
    & 1
  \end{tikzcd}
  \]
in which $\det(\phi_{\lambda,f})=\lambda$ for all $(\lambda,f)\in
\kk[x]\bowtie\kk^\times$, and which is split by the morphism
$\lambda\in\kk^\times\mapsto\phi_{0,\lambda}\in\Aut(A)$.

\end{thmlist}
\end{Theorem}

We have collected in this statement the results of
Propositions~\ref{prop:aut}, \ref{prop:ad-nilpotent} and~\ref{prop:lnd},
Corollaries~\ref{coro:aut-quot} and~\ref{coro:aut-0}. This information is
useful, for example, when computing the action of~$\Aut(A_N)$ on derived
objects, like Hochschild cohomology or cyclic homology, on which
exponentials of locally $\ad$-nilpotent elements tend to act trivially. For
now, let us say that such an explicit description of the automorphism group
allows us to compute its center easily. It turns out to be significant in
several ways:

\begin{Theorem}
There is a locally nilpotent derivation $\partial_0:A_N\to A_N$ such that
$\partial_0(x)=0$ and $\partial_0(y)=x^{N-1}$ such that the map
  \[ \label{eq:sigma-t}
  t \in \kk \mapsto \sigma_t\coloneqq\exp t\partial_0 \in\Aut(A_N)
  \]
is an injective $1$-parameter subgroup of~$\Aut(A_N)$ whose image is the
center of~$\Aut(A_N)$. 
\begin{thmlist}

\item The infinitesimal generator~$\partial_0$ of this $1$-parameter
subgroup is not inner and its class in~$\HH^1(A)$ spans the center of this
Lie algebra.

\item The element~$x$ of~$A$ is normal, in that $xA=Ax$, and the
automorphism $\sigma_1:A_N\to A_N$ is the automorphism associated to it, so
that $ax=x\sigma_1(a)$ for all $a\in A_N$.

\item The algebra $A_N$ is twisted Calabi--Yau of dimension~$2$, and the
automorphism $\sigma_1$ is its modular (or Nakayama) automorphism, so that in particular
there is an automorphism of $A_N$-bimodules
  \[
  \H^2(A_N,A_N\otimes A_N)
        = \Ext_{A_N^e}(A_N,A_N\otimes A_N)
        \to {}_{\sigma_1}A_N.
  \]

\item The derivation~$\partial_0$ preserves the canonical «order» filtration
on~$A_N$, so it induces a derivation $\overline\partial_0:\gr A_N\to\gr
A_N$ on the corresponding associated graded algebra. If we endow $\gr A$
with its standard Poisson structure coming from the commutator of~$A_N$,
then $\overline\partial_0$ is the modular derivation of~$A_N$ in the sense
of A.~Weinstein \cite{Weinstein}, and the corresponding modular flow
  \[
  \sigma:t\in\kk\mapsto\exp t\overline\partial_0\in\Aut(\gr A)
  \]
is exactly the $1$-parameter group of automorphisms induced by the
flow~\eqref{eq:sigma-t} above.

\end{thmlist}
\end{Theorem}

This theorem combines the results of Lemma~\ref{lemma:sigma-1},
Corollaried~\ref{coro:center} and~\ref{coro:expp0},
Proposition~\ref{prop:CY}, and Remark~\ref{rem:modular}. All the objects
mentioned in this theorem are canonical. For example, as the units of~$A_N$
are central, the modular automorphism of~$A_N$ as a twisted Calabi--Yau
algebra is well-determined. As we wrote in the theorem,
the derivation~$\partial_0$ is not inner, but it is «logarithmically
inner», in that it coincides with the restriction to~$A_N$ of the
derivation
  \[
  a\in (A_N)_x \mapsto \frac{1}{x}[x,a] \in (A_N)_x
  \]
of the localization~$(A_N)_x$ of~$A_N$ at its normal element~$x$.

\bigskip

A natural thing to do at this point is to describe 
the finite subgroups of $\Aut(A_N)$, and that is easy since we know the
group very well. Our Proposition~\ref{prop:finite-subgroups} implies, among
other things, the following:

\begin{Theorem}
Every finite subgroup of~$\Aut(A)$ is cyclic, and conjugated to the
subgroup generated by~$\phi_{0,\lambda}$, with $\lambda$ a root of unity
in~$\kk$. 
\end{Theorem}


Of course, with this result at hand the obvious next thing to do, following
the classics, is to describe the invariant subalgebras corresponding to the
finite subgroups of~$\Aut(A_N)$. This appears to be fairly difficult, and
we do not consider this problem here. We instead take a less classical
direction and try to extend the result of the theorem and find all actions
of finite dimensional Hopf algebras on~$A_N$ --- that is, to put it in a
colorful language, to find all \emph{quantum finite groups} of
automorphisms of~$A_N$. Now, at that level of generality we do not know how
to approach the problem, so we restrict ourselves to looking for all
actions of generalized Taft Hopf algebras on~$A_N$. We know that all finite
groups of automorphisms are cyclic, and generalized Taft algebras can be
viewed as «quantum thickenings» of cyclic groups, so this is a reasonable
first step. What we find is the following result.

\begin{Theorem}\label{thm:taft}
Let $n$ and~$m$ be integers such that $1<m$ and $m\mid n$, and let
$\lambda\in\kk^\times$ be a primitive $m$th root of unity in~$\kk$. There
is no inner-faithful action of the generalized Taft algebra
$T_n(\lambda,m)$ on~$A$.
\end{Theorem}

We refer to Section~\ref{sect:taft} of the paper --- which ends with a proof of this
theorem, stated there as Proposition~\ref{prop:taft-actions} --- for the
precise description of what we mean by generalized Taft algebra, and for the
definition of inner-faithfulness, which is due to T.~Banica and J.~Bichon \cite{BB}.
Let us just say here that inner-faithfulness intends to be to actions of
Hopf algebras what faithfulness is to actions of groups.

The negative nature of this theorem is disappointing, but not
unexpected. For example, this «no quantum finite symmetries» phenomenon
occurs on Weyl algebras, algebras of differential operators, and more
generally certain algebraic quantizations, as shown by J.~Cuadra,
P.~Etingof and C.~Walton in~\citelist{\cite{CEW:1}\cite{CEW:2}} and by
those last two authors in~\cite{EW}. There are many other classes of
algebras, though, which exhibit non-trivial quantum symmetries, and the
study of this is an extremely interesting line of work. In this direction,
one should mention the work of S.~Montgomery and H.-J.~Schneider \cite{MS}
on finite dimensional algebras generated by one element, extended
later by Z.~Cline~\cite{Cline}; L.~Centrone and F.~Yasumura \cite{CY} on
finite dimensional algebras; Y.~Bahturin and S.~Montgomery \cite{BM} on
matrix algebras; R.~Kinser and C.~Walton \cite{KW} on path algebras;
J.~Gaddis, R.~Won and D.~Yee \cite{GWY} on quantum planes and quantum Weyl
algebras; Z.~Cline and J.~Gaddis \cite{CG} on quantum affine spaces,
quantum matrices, and quantized Weyl algebras; and J.~Gaddis and R.~Won
\cite{GW} in quantum generalized Weyl algebras.

The way we prove Theorem~\ref{thm:taft} is by studying the twisted
derivations of our algebra~$A_N$. This is a natural approach at this point,
for it consists of fixing an automorphism $\phi$ of the algebra and
computing $\HH^1(A_N,{}_\phi A_N)$, the first Hochschild cohomology space
of the algebra~$A_N$ with values in the $A_N$-bimodule that can be obtained
from~$A_N$ by twisted the left action by the automorphism~$\phi$, and we
can more or less use the same ideas that we used to compute the regular
Hochschild cohomology of~$A_N$. We do this in Section~\ref{sect:twist}. We
do not describe here the precise results obtained therein because of their
rather technical nature. We would like, nonetheless, to direct the reader's
attention to our Remark~\ref{rem:general-nonsense}, which explains the way
we construct twisted derivations in terms of the Gerstenhaber algebra structure
of Hochschild cohomology, and which is probably of more general applicability.
Finally, it should be remarked that the study of twisted Hochschild
cohomology should be useful to study actions of general finite dimensional
Hopf algebras, since we know from the classification theory of
A.~Andruskiewitsch and H.-J.~Schneider~\cite{ASch}  that finite dimensional
Hopf algebras are in many cases «built» from group-like and skew-primitive
elements, which give rise to automorphisms and twisted derivations of the
algebras upon which they act.

\bigskip

All the work described above involves the first Hochschild cohomology Lie
algebra~$\HH^1(A_N)$ and the automorphism group~$\Aut(A_N)$. Our algebra
also has a non-zero second Hochschild cohomology space $\HH^2(A_N)$, and
according to the classical deformation theory of M.~Gerstenhaber
\cite{Gerstenhaber:deformation}, its elements produce «by integration»
formal deformations of the algebra~$A_N$ just as the elements of~$\HH^1(A)$
produce (more or less\dots) automorphisms by exponentiation --- indeed, as
$\HH^3(A)=0$ this procedure of integration of elements of~$\HH^2(A)$ is
unobstructed, so always possible. We leave for future work the explicit
construction of these formal deformations, in the same vein as we
constructed explicitly derivations spanning $\HH^1(A)$. 

The natural thing to do, actually, is to study this problem for the whole
family of algebras~$A_h$ of~\citelist{\cite{BLO:1}\cite{BLO:2}\cite{BLO:3}}
to which the algebra~$A_N$ belongs at the same time: these algebras deform,
under appropriate hypotheses, to one another, and they all appear as
deformations of~$A_N$. The algebra~$A_N$ is the most singular element of
the class, so the geometry of the family at that point is the central point
to elucidate. This was the main motivation for the study of the
automorphism group of~$A_N$ described above, in fact. The description of
the Lie action of~$\HH^1(A_N)$ on~$\HH^2(A_N)$ that can be read off from
the work~\cite{LSo} of S.~Lopes and A.~Solotar should also be important to
do this.

\bigskip

Let us finish this introduction with a question. The form of the
derivations of~$A_N$ that we find in Section~\ref{sect:explicit} and, in
particular, the observations made at the end of Section~\ref{sect:phi}
about the elements~$\Phi_j$, hint at the idea that the algebra~$A_1$ that we
get by putting $N=1$ is not the correct «limit as $N$ converges to~$1$ of
the algebra~$A_N$». 

\begin{Question}
Is there an algebra~$\tilde A_1$ which better reflects the result of
«taking the limit of~$A_N$ as $N$ goes to~$1$» in that it has exterior
derivations involving the Bernoulli polynomials that we found in
equation~\eqref{eq:bern} at the end of Section~\ref{sect:phi} and the
Faulhaber formula?
\end{Question}

We could say that the family $(A_N)_{N\geq1}$ is, in a not very precise sense, flat with
respect to~$N$, but our calculation of~$\HH^1(A_N)$ shows that this space
is constant \emph{except} at~$N=1$. An algebra~$\tilde A_1$, obtained by
something like «dimensional regularization» but with respect to~$N$, would
fix this. Another example of the anomalous status of the algebra~$A_1$ in
the family $(A_N)_{N\geq1}$ is provided by Remark~\ref{rem:laguerre} below.
