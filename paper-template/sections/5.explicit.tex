\section{Explicit representatives for classes in
\texorpdfstring{$\HH^1(A)$}{HH1(A)}}
\label{sect:explicit}

In this section, as promised, we will give explicit derivations
representing the elements of~$\HH^1(A)$ when $N\geq2$. We start with a
simple observation that allows us to show that some derivations are not
inner.

\begin{Lemma}\label{lemma:not-inner}
Let $d:A\to A$ be a derivation.
\begin{thmlist}

\item If $d(x)=0$, then $d(y)\in\kk[x]$.

\item If $d$ is inner, then $d(x)\in x^NA$. If additionally $d(x)=0$, then
also $d(y)\in x^N\kk[x]$.

\end{thmlist}
\end{Lemma}

\begin{proof}
\thmitem{1} If $d(x)=0$, then $0 = d([y,x]-x^N) = [d(y),x]$, and $d(y)$ is
in the centralizer of~$x$ in~$A$, which we know to be equal to~$\kk[x]$.

\thmitem{1} If $d$ is inner, then there is an $u\in A$ such
that $d(a)=[u,a]$ for all $a\in A$ and, in particular, $d(x)=[u,x]\in
[A,x]=x^NA$. If additionally $d(x)=0$, then $[u,x]=0$ and therefore
$u\in\kk[x]$: it follows from this that $d(y)=[u,y]=x^Nu'\in  x^N\kk[x]$.
\end{proof}

The simplest derivations of our algebra are those of non-positive degree.
We describe them in the following result.

\begin{Proposition}\label{prop:hh1-low}
Suppose that $N\geq2$. For each $l\in\inter{-N+1,0}$ there is a derivation
$\partial_l:A\to A$ of degree~$l$ such that
  \begin{align}
  \partial_l(x) &= 0, 
  &
  \partial_l(y) &= x^{l+N-1},
\intertext{and there is a derivation $E:A\to A$ of degree~$0$ such that}
  E(x) &= x, 
  &
  E(y) &= (N-1)y.
  \end{align}
For each $l\in\inter{-N+1,-1}$ the space~$\HH^1(A)_l$ is freely
spanned by the class of~$\partial_l$, and $\HH^1(A)_0$ is freely spanned
by~$\partial_0$ and~$E$. 
\end{Proposition}

The derivation~$\partial_0$ is the same one we encountered in
Corollary~\ref{coro:expp0}.

\begin{proof}
A very simple calculation shows that there are indeed derivations of~$A$ as
described in the statement of the proposition, and they manifestly have the
degrees given there. None of the derivations
$\partial_{-N+1}$,~\dots,~$\partial_0$ and~$E$ is inner ---~this follows
immediately from the second part of Lemma~\ref{lemma:not-inner}, because
$N\geq2$~--- and then, in view of Proposition~\ref{prop:hh1-series}, we see
that second claim of the proposition holds.
\end{proof}

To deal with classes in~$\HH^1(A)$ of positive degree, we start by showing
that they have a representative with a useful normalization:

\begin{Lemma}\label{lemma:normalization}
Suppose that $N\geq2$. Let $l$ be a positive integer, and let $i$ and $j$
be the integers such that $l+1=i+j(N-1)$, $j\geq0$, and $1\leq i\leq
N-1$. There exists a derivation $d:A\to A$ that is homogeneous of
degree~$l$, is not inner, and has $d(x)  = x^iy^j$.
\end{Lemma}

\begin{proof}
Let $l$ be a positive integer. The vector space $\HH^1(A)_l$ has
dimension~$1$. We pick a derivation~$d:A\to A$ homogeneous of degree~$l$
whose cohomology class spans that vector space. The element~$d(x)$ of~$A$
is then homogeneous of degree~$l+1$, and we can write it in the form
$u+x^Nv$ with $u$ an homogeneous element of
$P\coloneqq\sum_{k=0}^{N-1}x^k\kk[y]$ of degree~$l+1$, and $v$ a
homogeneous element of~$A$ of degree~$l+1-N$. There is a $w\in A$ such that
$[w,x]=x^Nv$, and it can be taken to be homogeneous of degree~$l$: the
derivation $d'\coloneqq d-\ad(w)$ is then homogeneous of degree~$l$,
cohomologous to~$d$, and has $d'(x)\in P$. The upshot of all this is that
we could have simply chosen our original derivation~$d$ so that $d(x)$ is
in~$P$ from the start, and we do so now.

An homogeneous element of degree~$l+1$ in~$P$, such as $d(x)$, is a linear
combination of the monomials of the form $x^ry^s$ with $r+s(N-1)=l+1$ and
$0\leq r<N$. We consider now two cases.
\begin{itemize}

\item Suppose first that $N-1\nmid l+1$, so that in each such monomial we
have in fact that $1\leq r<N-1$, and therefore that~$r$ and~$s$ are
necessarily equal to~$i$ and~$j$, respectively: in particular, there is
exactly one such monomial, $x^iy^j$, and there is thus a scalar~$\alpha$
such that $d(x)=\alpha x^iy^j$.

\item Suppose next that $N-1\mid l+1$, so that $i=N-1$. There are
now two monomials of degree~$l+1$ in~$P$, namely $x^{N-1}y^{j}$ and
$y^{j+1}$, and there are scalars~$\alpha$ and~$\beta$ such that $d(x) = \alpha
x^{N-1}y^{j} + \beta y^{j+1}$. As
  \begin{align}
  0 &= d([y,x]-x^N)
     = [d(y),x] + [y,d(x)] - \sum_{s+1+t=N}x^sd(x)x^t \\
    &= [d(y),x] + \alpha(N-1)x^{2N-2}y^{j} 
         - \alpha \sum_{s+1+t=N}x^{s+N-1}y^{j}x^t  
         - \beta\sum_{s+1+t=N}x^sy^{j+1}x^t
       \\
    &\equiv -\beta Nx^{N-1}y^{j+1} \mod(F_{j}+x^NA),
  \end{align}
the scalar~$\beta$ is actually~$0$.
\end{itemize}
The end result of all this is that in any case there is a scalar~$\alpha$
such that $d(x) = \alpha x^iy^j$. If $\alpha$ is~$0$, then the first part
of Lemma~\ref{lemma:not-inner} tells us that $d(y)\in\kk[x]$ and, since
$d(y)$ is a homogeneous element of degree~$l+N-1$, that in fact there is a
scalar~$\gamma$ such that $d(y)=\gamma x^{l+N-1}$: this is impossible,
since in that case we have $d=\gamma\ad(x^l)$, and the derivation~$d$ is
not inner. The derivation $\alpha^{-1}d$ thus satisfies the condition we
want.
\end{proof}

With the same idea that we used in the end of this proof we can also obtain
the following criterion that allows us to prove a derivation is inner:

\begin{Lemma}\label{lemma:cut}
Suppose that $N\geq2$. Let $l$ be a positive integer, and let $i$ and $j$
be the integers such that $l+1=i+j(N-1)$, $j\geq0$, and $1\leq i\leq N-1$.
A homogeneous derivation $d:A\to A$ of degree~$l$ such that $d(x)\in
F_{j-1}$ is inner.
\end{Lemma}

\begin{proof}
Let $d:A\to A$ be a homogeneous derivation of degree~$l$ such that $d(x)$
belongs to~$F_{j-1}$. There are then scalars $a_1$,~\dots,~$a_j$ in~$\kk$
such that $d(x)=\sum_{k=1}^ja_kx^{i+k(N-1)}y^{j-k}$ and therefore $d(x)\in
x^NA$. As we know, this implies that there is an element~$u$ in~$A$, which
can be chosen of degree~$l$, such that $d(x)=[u,x]$, and therefore the
derivation $d'\coloneqq d-\ad(u)$ is homogeneous of degree~$l$ and vanishes
on~$x$. According to the first part of Lemma~\ref{lemma:not-inner}, we have
$d'(y)\in k[x]$, so that in view of the homogeneity of~$d'$ there is a
scalar~$b$ in~$\kk$ such that $d'(y)=bx^{l+N-1}=-\frac{b}{l}[x^{l},y]$.
It follows from this that $d=\ad(u)-\frac{b}{l}\ad(x^{l})$, and this
proves the lemma.
\end{proof}

We need yet one more commutation identity. We promise it is the last one.

\begin{Lemma}\label{lemma:identity}
For each element $u$ of $A$ we have that
  \begin{equation*}
  \sum_{s+1+t=N}x^sux^t - Nx^{N-1}u
        = \left[
          \sum_{s+2+t=N}(s+1)x^sux^t,x
          \right].
  \end{equation*}
\end{Lemma}

\begin{proof}
The identity can be proved by expanding the commutators
appearing in its right hand side and simplifying.
\end{proof}

Using all these results, we can finally exhibit representatives for classes
in~$\HH^1(A)$:

\begin{Proposition}\label{prop:hh1-high}
Suppose that $N\geq2$. Let $l$ be a positive integer, and let $i$ and~$j$
be the integers such that $l+1=i+j(N-1)$, $j\geq0$, and $1\leq i\leq N-1$.
The vector space~$\HH^1(A)_l$ is spanned by a derivation $\partial_l:A\to
A$ that is homogeneous of degree~$l$ and such that
  \[
  \partial_l(x) = x^iy^j,
  \qquad
  \partial_l(y) = \sum_{s+2+t=N}(s+1)x^{s+i}y^jx^t + (N-i)x^{i-1}\Phi_{j+1}.
  \]
\end{Proposition}

The element~$\Phi_j$ appearing here is the one from
Proposition~\ref{prop:phi} of Section~\ref{sect:phi}. In this proposition
the integer~$l$ is assumed to be positive: we can observe that if we allow
it to be~$0$, then the formulas in the statement actually do also produce a
derivation of degree~$0$ that maps~$x$ and~$y$ to~$x$ and to
$N(N-1)x^{N+1}/2+(N-1)y$, and that this is the derivation
$E+N(N-1)\partial_0/2$ that we have already found in
Proposition~\ref{prop:hh1-low}.

\begin{proof}
Let $d:A\to A$ be a derivation that is homogeneous of degree~$l$
and not inner, and that satisfies the condition of
Lemma~\ref{lemma:normalization}, so that $d(x)=x^iy^j$. As $[y,x]-x^N=0$
in~$A$, we have, using the result of Lemma~\ref{lemma:identity}, that
  \begin{align}
  [d(y),x] 
        &= \sum_{s+1+t=N}x^sd(x)x^t - [y,d(x)]  \\
        &= \left[
           \sum_{s+2+t=N}(s+1)x^{s+i}y^jx^t,x
           \right] 
           + (N-i)x^{i+N-1}y^j,
  \end{align}
and therefore that
  \[
  \left[
  d(y) - \sum_{s+2+t=N}(s+1)x^{s+i}y^jx^t - (N-i)x^{i-1}\Phi_{j+1},
  x
  \right]
  = 0.
  \]
If follows from this that there exists an $f\in\kk[x]$ such that
  \[
  d(y) = \sum_{s+2+t=N}(s+1)x^{s+i}y^jx^t + (N-i)x^{i-1}\Phi_{j+1}+ f.
  \]
Clearly $f$ has to be homogeneous of degree~$l+N-1$, so equal to
$(l+N-1)\lambda x^{l+N-1}$ for some~$\lambda\in\kk$. The derivation
$\partial_l\coloneqq d-\lambda\ad(x^{l-1}y)$, which is cohomologous to~$d$,
is as in the statement of the theorem.
\end{proof}

We remark the nice fact that we were able exhibit the derivation
in~Proposition~\ref{prop:hh1-high} without needing to prove that it
actually is a derivation ---~which is probably a very unpleasant
calculation! 
